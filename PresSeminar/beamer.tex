\documentclass{beamer}

\usepackage[utf8]{inputenc}
\usepackage[english]{babel}
\usepackage[T1]{fontenc}

%\usepackage{multicol}

\title{PAC learning for gene networks}
\date{20 Apr. 2017}
\author{Arthur \textsc{Carcano}\inst{1} \and François \textsc{Fages}\inst{2} \and Sylvain \textsc{Soliman}\inst{2}}
\institute{\inst{1}\'{E}cole Normale Supérieure %
	\and \inst{2}Inria, Lifeware group}

\usetheme[block=fill,progressbar=frametitle,sectionpage=simple]{metropolis}

\usepackage{color}
\usepackage{subcaption}
\usepackage{adjustbox}
\usepackage{algpseudocode}
\usepackage{array}

\usepackage{verbatim}

\newcommand{\transition}{\vspace{1em}\flushright \itshape}
\newcommand{\ok}{\textcolor{blue}{\checkmark}}
\newcommand{\nope}{\textcolor{red}{$\times$}}
\newcommand{\B}{\mathbb{B}}
\begin{document}
	
	\frame{
		\titlepage
	}
	
	\frame{
		\frametitle{Outline of the talk}
	\tableofcontents
}

\section{What PAC learning is about}
\subsection{First intuition}
\begin{frame}{First intuition}
\begin{itemize}
	\item There is a function $F$
	\begin{itemize}
		\item Input: $n$ boolean
		\item Output: one boolean
	\end{itemize}
	\item We want to recover $F$
	\item We can:
	\begin{itemize}
		\item Get \textbf{Positive examples} (values for which $F$ is true): \textbf{cheap} (sample)
		\item \textbf{Evaluate} $F$ on some input: \textbf{expensive} (oracle)
	\end{itemize}
\end{itemize}	
\transition Can we get $F$ back ?
\end{frame}
\begin{frame}{Saving private $F$}
	\begin{itemize}
		\item In general, we cannot get $F$ back without $2^n$ calls to the oracle
		\item But with \textbf{approximation} and \textbf{restriction} on $F$ we can.	
	\end{itemize}
\end{frame}

\subsection{A bit of math}
\begin{frame}{A bit of math}
	From now on we fix $n \in \mathbb{N}$.
	\begin{block}{Vector}
	An element $v$ of $\mathbb{B}^n = \{0,1\}^n$
	\vspace{1em}
	
	$\sim$ A possible assignment for $n$ boolean variables
	\end{block}
\begin{block}{Boolean function}
	$F : \mathbb{B}^n \longrightarrow \mathbb{B}$
\end{block}
\end{frame}
\begin{frame}{Learnable classes}
\begin{block}{Learnable class}
		Set of boolean functions s.t.:
		\vspace{1em}
		
		$\exists$ An algorithm that runs in time polynomial in $n$ and $h$, that only calls \textsc{Sample} and \textsc{Oracle}
		\vspace{1em} and outputs $G$
		
		\begin{itemize}
			\item $G(v) = 1 \Rightarrow F(v) = 1$ (no false positive)
			\item Probability of sampling a false negative ($F(v) = 1, G(v)=0$) is less than $h^{-1}$.
		\end{itemize}
\end{block}
\end{frame}
\begin{frame}{Two classes of interest}
Two learnable class on which we focus
\begin{enumerate}
	\item Monotone DNF (with no negation)
	\item $k$-CNF (clauses of at most $k$ terms)
\end{enumerate}
\end{frame}
\subsection{Intuition of the link with biology}
\begin{frame}{Link with biology (intuition)}
	\begin{itemize}
		\item A gene is activated or not (boolean)
		\item There is $n$ genes  (vector)
		\item The (de)activation of a gene depends on the state (activated or not) of the others (boolean function)
	\end{itemize}
\centering
\slshape
Can we learn these (de)activation functions ?
\end{frame}
\section{Monotone DNF}
\subsection{Valiant's result}
\begin{frame}{Monotone DNF}
	\begin{block}{Monotone DNF}
		\begin{description}
			\item[DNF] Disjunction of Conjunctions
			\item[Monotone] No negative terms
		\end{description}
	\end{block}
\begin{block}{Learnability of monotone DNFs}
	Monotone DNF are learnable with calls to \textsc{Oracle} and \textsc{Sample}
\end{block}
\transition What does it mean from the biologist PoV ?
\end{frame}
\subsection{Biology vs PAC learning}
\begin{frame}{Biology vs PAC learning : \textit{sampling} (1/2)}
\begin{figure}
\[
\underset{\vspace{1em}(a)}{
\left(\begin{array}{c}
0\\ 1\\ 0\\ 1
\end{array}\right)}
\rightarrow
\underset{\vspace{1em}(b)}{
	\left(\begin{array}{c}
	\textbf{1}\\ 1\\ 0\\ 1
	\end{array}\right)}
\rightarrow
\underset{\vspace{1em}(c)}{
	\left(\begin{array}{c}
	1\\ 1\\ 0\\ \textbf{0}
	\end{array}\right)}
\]
\caption{Three step of a trace}
\end{figure}
\begin{itemize}
	\item $(a)$ is a positive example for the activation of the first gene
	\item $(b)$ is a positive example for the deactivation of the last gene
\end{itemize}
\end{frame}
\begin{frame}{Biology vs PAC learning : \textit{sampling} (1/2)}
\begin{itemize}
	\item Each step is a positive example for \textbf{one} of the (de)activation functions
	\item Get enough samples for each of the $2n$ (de)activation functions (this is the number of time we have to call \textsc{Sample} !)
\end{itemize}
\end{frame}
\begin{frame}{Biology vs PAC learning : \textit{oracle} (2/2)}
To evaluate an (de)activation function we need
\begin{enumerate}
	\item To be able to set the system in a given state (\textbf{hard !})
	\item To observe a step (as hard as for sampling)
\end{enumerate}
\pause
And do this an \textbf{infinite number of times} (so that if we didn't see a transition, it is because it can only happen with probability 0)
\end{frame}
\section{$k$-CNF}
\end{document}