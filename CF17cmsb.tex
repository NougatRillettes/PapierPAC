\documentclass{llncs}
\usepackage{amsmath,amssymb,amsfonts,stmaryrd}
\usepackage{graphicx}
\usepackage[usenames,dvipsnames]{color}
\usepackage{listings}
\lstset{
   basicstyle=\scriptsize\ttfamily,
%   frame=single,
%   breaklines=true,
}
\usepackage{hyperref}
\usepackage{tikz}
\usetikzlibrary{arrows}
\usepackage{algorithm}
\usepackage{algpseudocode}
\usepackage[T1]{fontenc}
\usepackage{verbatim}

%%%%%
\newcommand{\wip}[1]{\textcolor{Purple}{WIPWIPWIPWIP #1 WIPWIPWIPWIP}}
\newcommand{\francois}[1]{\textcolor{blue}{#1}}


\newcommand\for{\mathbin{\text{\it for}}}
\newcommand\inhibits{\mathbin{-\!\Yleft}}
\newcommand\activates{\rightarrow}
\newcommand{\I}[1]{\mathit{#1}}

\newcommand{\ra}{\rightarrow}
\newcommand{\lra}{\longrightarrow}

\DeclareMathOperator{\Set}{Set}


\newcommand{\defleq}{\sqsubseteq_{\text{def}}}

\begin{document}
\bibliographystyle{splncs03}


\title{Probably Approximately Correct Learning of Regulatory Networks from Boolean Traces}

\author{Arthur Carcano\inst{1} \and Fran\c{c}ois Fages\inst{2}}

\institute{Ecole Normale Sup\'erieure, Paris, France\\
  \email{arthur.carcano@ens.fr}
\and Inria, University Paris-Saclay, Lifeware group, France\\
   \email{Francois.Fages@inria.fr} 
}
\maketitle



\begin{abstract}

\end{abstract}

\section{Introduction}


Biological modeling is still an art which is currently limited in its applications by the number of available modelers
Automating the process of model building is thus a very desirable goal
to attack new applications, develop patient-tailored therapeutics,
and also design experiments that can now be largely automated
with a gain in both the quantification and the reliability of the observations, at both the single cell and cell population levels.

Machine learning is revolutionizing the statistical methods in biological data analytics,
data classification and clustering, and for making predictions from static measurements.
However, learning dynamical models from temporal data is more challenging.
There has been early work on the use of machine learning techniques, such as inductive
 logic programming \cite{Muggleton95ngc} combined with active learning in the vision of the ``robot scientist'',
to infer gene functions \cite{BMOKRK01etai},
metabolic pathway descriptions \cite{AM02etai,AM02slps}
or gene influence systems \cite{BCRG04jtb},
or to revise a reaction model with respect to CTL properties \cite{CCFS06tcsb}.
Since a few years, progress in this field can be measured on public benchmarks
of the ``Dream Challenge'' competition \cite{Meyer14bmc}.
In this fastly moving field, we focus here on the general purpose learning framework of Leslie Valliant for learning the structure of a mechanistic model.

We show...

\section{Preliminaries on PAC Learning}\label{pac}

In his seminal paper on a theory of the learnable \cite{Valiant84cacm},
Valiant questioned what can be learned from a computational viewpoint,
and introduced the concept of probably approximate correct (PAC) learning,
together with a general-purpose polynomial-time learning protocol.
Beyond the learning algorithms that one can derive with this methodology,
Valiant's theory of the learnable has profound implications
on the nature of biological and cognitive processes,
of collective and individual behaviors,
and on the study of their evolution \cite{Valiant13book}.
In this section, we simply recall the general theory of PAC learning,
and illustrate it with the learning of Boolean gene networks from gene expression data.

The learning protocol for Boolean functions considers
a finite set of Boolean variables $x_1,\ldots,x_s$.
A vector is an assignment of the $s$ variables to $\{0,1,*\}$
where the symbol $*$ denotes the undetermined.
A vector is total if it contains no undetermined value.
A Boolean function $F:\{0,1\}^s \rightarrow\{0,1\}$
assigns a Boolean value to each total vector.

We denote by $\defleq$ the definedness order, that is, for any positive integer $n$, and any vectors $v$ and $w$ in $\{0,1,*\}^n$, we have $v \defleq w$ if and only if for all $i$, $1 \leq i \leq n$, we have $v_i = *$ or $v_i = w_i$.

Then, we define a boolean concept $C:\{0,1,*\}^s \rightarrow\{0,1,*\}$ as a function that
\francois{Attention chez Valiant je lis qu'un concept est une fonction dans $\{0,1\}$ pas $\{0,1,*\}$, voir article de 1984 section 2 ajoute au git. Cependant en section 3 les ''programmes'' a apprendre peuvent rendre indetermine en resultat.}
assigns a Boolean value to non total vectors,
with the following independence constraint:
for any vector $v$ and $w$, if $v \defleq w$ then $C(v) \defleq C(w)$.

The PAC learning protocol considers a hidden Boolean function $F$,
a class $\cal M$ of models to learn, $f(x_1,...,x_s) \in \{0,1,*\}$,\wip{(weird formulation I don't get what it is introducing)}
\francois{Heu j'avoue ne pas arriver a me relire. A reprendre}
a set of positive examples, i.e.~a set of vectors $v$ for which $F(v)=1$,
and an arbitrary probability distribution $D$ over this set
for representing the relative frequency of the positive examples.
The restriction to positive examples is for the sake of simplicity.
\wip{(How could we add them ? What would be the pros and cons ?)}
\francois{Valiant l'evoque dans la section 3 en haut de la page 1137. On s'interdit de ne pas reproduire les positifs mais on pourrait aussi s'interdire de ne pas produire des negatifs ce que l'on ne fait pas par simplicite. Sinon il faut doubler la representation. Mais c'est peut etre ce que tu fais avec la fonction qui apprend 1 et celle qui apprend 0 ?}
The PAC learning protocol then allows for
\begin{itemize}
  \item
calls for positive examples, i.e.~vectors $v$ such that $F(v)=1$ given with probability $D(v)$,
  \item
calls for oracle on some input $v$ to know the value of $F(v)$
\end{itemize}


%\begin{definition}[\cite{Valiant84cacm}]
A class $\cal M$ of models is learnable in a given learning protocol, if there exists an algorithm $\cal A$ such that:
\begin{itemize}
  \item
$\cal A$ runs in polynomial time in $s$ -- the size of the models to learn -- and $h$ the precision parameter.
  \item
For all models $f$ in $\cal M$,
all vector distributions $D$ on which $f$ outputs $1$,
$\cal A$ deduces with probability $\ge 1-h^{-1}$ a model $g$ in $\cal M$ such that
\begin{itemize}
  \item
$g(v)=1$ implies $f(v)=1$
\item
$\sum_{v\ s.t.~f(v)=1\ g(v)\neg=1} D(v) < h^{-1}$
\end{itemize}
\end{itemize}
%\end{definition}


Interestingly, Valiant showed the learnability of some important classes of functions in this framework,
in particular for Boolean formulae in conjunctive normal forms with at most $k$ literals (k-CNF)
and for monotone (i.e.~negation free) Boolean formulae in disjunctive normal form (DNF).
The computational complexity of the PAC learning algorithms for these classes of functions is expressed in terms of the function
$L(h,S)$ defined as the smallest integer $i$ such that
in $i$ independent Bernoulli trials, each with probability at least $h^{-1}$ of success, the probability of having fewer than $S$ successes is less than $h^{-1}$.
Interestingly, this function is quasi-linear in $h$ and $S$, i.e.~for all integers $S\ge 1$ and reals $h>1$, $L(h,S) \le 2h(S+log_eh)$.

%\begin{theorem}[
First, for any $k$, the class of k-CNF formulae is learnable with an
algorithm that uses $L(h,(2s)^{k+1})$ examples and no oracle \cite{Valiant84cacm}.
%\end{theorem}
The algorithm used in the proof proceeds as follows
\begin{enumerate}
  \item Initialize $g$ to the conjunction of all possible $(2s)^{k+1}$ disjunctions of at most $k$ literals,
\item Call $L(h,(2t)k+1)$ positive examples $v$,
\item Delete all the disjunctions in $g$ that do not contain a literal true in $v$.
\end{enumerate}


%\begin{theorem}[\cite{Valiant84cacm}]
    Second, the class of monotone DNF formulae is also learnable with an algorithm that uses $L(h,d)$ examples and $ds$ calls to the oracle,
    where $d$ is the largest number of prime implicants in an equivalent prime DNF formula \cite{Valiant84cacm}.
%\end{theorem}
The algorithm is the following:
\begin{enumerate}
\item Initialize $g$ with constant zero,
\item
Do $L(h,d)$ calls to positive examples $v$,
\item
If $g$ is not implied by $v$, add the conjunction of determined literals that are essential to $f$ which is determined by $ds$ calls to the oracle.
\end{enumerate}

\section{PAC Learning Thomas's Gene regulatory Networks}

\subsection{$k$-CNF Models of Thomas's Networks}

  This Boolean framework perfectly fits the Boolean semantics of Thomas's gene regulatory networks.
  Indeed in that formalism, given a network of $s$ genes $x_1,\ldots,x_s$, and an integer $1 \leq k \leq s$ we can define ${x_k}_+$ (resp. ${x_k}_-$): $\{0,1\}^s \rightarrow\{0,1\}$ the activation (resp. deactivation) function of $x_k$.
  
  These Boolean functions are best represented by Boolean concepts in PAC terminology
  in order to make explicit the independent genes.
  Then, the problem of building such a Boolean model \emph{\`a la} Thomas of gene activation is to give for each gene
  two Boolean activation and deactivation functions that are compatible with the observed temporal data of gene activation.
It is worth noticing that the PAC learning protocol makes it possible to learn such Boolean models of gene regulation
not only from a given finite set of positive gene activation observations,
but also from new biological experiments designed by the PAC learning algorithm itself.

\subsection{PAC-Learning from Positive Examples}

$k$-CNF formulae can be used to represent Thomas's gene regulatory network functions with some reasonable restrictions on their connectivity.
In this case, the algorithm is repeated $2s$ times for learning each gene (de-)activation function.
In this representations, the initialization of the learned function $g$ to the false constraint expressed as the conjunction of all possible disjunctions
leads to the learning of a minimal generalization of the positive examples.\wip{I don't really get this part}
\francois{Is it better ?}

\section{Our implementation}
Our implementation represents the lattice of $k$-clauses ordered by implication. Our data structure allows:
\begin{itemize}
	\item $O(1)$ access to any $k$-clause
	\item From clause $a$, $O(1)$ access to the smallest clauses implied by $a$ and to the biggest clauses that implies $a$
\end{itemize}

For the sake of evaluating the learning algorithm, 
we simulate wet lab experiments \emph{in silico}, using Gillespie's algorithm for stochastic simulation. 
This stochastic semantics gave us traces in $\mathbb{N}^s$ that we abstract as traces in $\{0,1\}^s$.
\francois{Ce sera un point sur lequel revenir en discussion, a priori ca serait mieux d'apprendre la discretisation}

Our simulator uses a custom language to specify experiments and allows us to set the amount of each species before the experiment is run. The number of run is given by Valiant's bound on the number of sampling calls.


\section{Results and comments}

We ran three example reactions, given in figures \ref{test}, \ref{preypred}, and \ref{lympho}.
\begin{figure}[htbp]
	\verbatiminput{examples/test.reac}
	\vspace{-1em}
	\caption{A test reaction, A and E appear naturally in the medium, and A can be turned into B in absence of E. B  can be turned into C.\label{test}}
\end{figure}
\begin{figure}[htbp]
	\verbatiminput{examples/Lokta.reac}
	\vspace{-1em}
	\caption{A Prey-Predator model. Only predators die of old age.\label{preypred}}
\end{figure}
\begin{figure}[htbp]
	\verbatiminput{examples/lymphocyte.reac}
	\vspace{-1em}
	\caption{The Th lymphocyte differentiation model.\label{lympho}}
\end{figure}

The first example gave perfect results, as indicated on figure \ref{test_res}.

Output is to be read as follow:\\
\texttt{Foo+:~[['A','Z'],~['!E'],~['!Foo']]} means that the activation (\texttt{+}) function of B is $(A \vee Z)\wedge\neg E$. Remark that to be activated, Foo obviously needed to be deactivated first.

The second one (figure \ref{preypred_res}) displays two typical error: because there more than 1 prey, in the Gillespie algorithm, the eating of a prey is way more likely than the death of old age of a predator, hence the deactivation (ie extinction) of predators happen only when they cannot eat anymore and the only possible reaction is the third: death. This lead the Gillespie algorithm to believe that the absence of prey is needed for extinction.

Finally the third one (figure \ref{lympho_res}) gives very rough results and honesty forces us to admit that empiric tweaking of the model has been done to get results that were not too bad.

\begin{figure}
	\verbatiminput{examples/test.result}
	\caption{Results for the test example \label{test_res}}
\end{figure}
\begin{figure}
	\verbatiminput{examples/lokta.result}
	\caption{Results for the Prey-Predator model\label{preypred_res}}
\end{figure}
\begin{figure}
	\verbatiminput{examples/lympho_edited.result}
	\caption{Results for lymphocite model, edited. Lines with MISS are one on which the algorithm is too far. The tautological negation (\texttt{!A} for \texttt{A+}) have also been deleted.\label{lympho_res}}
\end{figure}


\section{Active Learning of Positive Influence Systems}

The (positive) Boolean semantics of biochemical influence systems described in Section \ref{semI}
can be directly represented by the disjunction of the (positive) enabling conditions of each, either positive or negative, influence on a given target,
i.e.~by a monotone DNF formula for each activation or inhibition of each target.
In the Lotka-Volterra influence system of Example \ref{LVi}, the algorithm above is thus expected to learn the structure of the influence system
(without the stochiometry of course),
from the observation that the prey can disappear only in presence of the predator
while the predator can always disappear in presence or absence of the prey.

  Learning reaction models from observed transitions is much more tricky,
  since some reactions may change the Boolean value of several reactants or products in one single transition.
%  Let us first remark that the transition relation $r(x_1,\ldots,x_s,x'_1,\ldots,x'_s)$
%  is naturally represented in DNF by the disjunction of the conjunctions associated to each reaction
%  (remember that there can be several Boolean transitions associated to one reaction for taking
%into account the possibly partial or total consumption of the reactants).
%  is monotonic in the predecessor variables $x_i$'s in the postive Boolean semantics of reactions.
%  Furthermore it is also monotonic in the successor variables $x'_j$ since we consider
%  all partial or total consumptions of reactants.
  Therefore, it is not only the activation and inhibition functions of each species which are to be learnt,
  but the update functions of pairs and triples of species if we restrict to elementary reactions with at most two reactants or products.
  In this case, the update functions can be represented by monotonic DNF formulae, since the (positive) Boolean semantics of a reaction system does not test the absence. 
Furthermore,   one cannot expect to learn the structure of such a reaction network
from the observation of the state transitions from one single initial state.
The learning algorithms assumes that the positive examples of the state transition relation be distributed
among the whole vector space.
For instance, in the MAPK example \ref{MAPK}, in addition to the initial state of the wild type organism where all the kinases and phosphatases are present,
it is necessary to consider some mutated organisms, in which some kinases or phosphatases are absent,
in order to gain information on the precise conditions of activation and deactivation of the different forms of the kinases.
This strategy is essentially similar to what the biologists do to elucidate the structure of biological processes
in a qualitative manner.


\section{Related Work}

Logic Programming, and especially \emph{Answer Set Programming} (ASP), provide particularly efficient tools such as CLASP \cite{GKNS07lpnmr} to develop learning algorithms for Boolean models.
They were applied in \cite{GSTUV08iclp} to detect inconsistencies in large biological networks,
and have been subsequentially applied to the inference of gene networks from gene expression data.

Interestingly, ASP has also been combined with CTL model-checking in \cite{OPSSG16biosystems} to learn mammalian signalling networks from time series data, 
and identify erroneous time-points in the data, a possibility not considered in the previous presentation of PAC learning.


Budgeted learning extends active learning with a notion of cost for the calls to the oracle.
The original motivation for the budgeted learning protocol came from medical applications in which the outcome of a treatment,
drug trial, or control group is known, and the results of running medical tests are each available for a price \cite{DZBSM13ml}.
In this context, multi-armed bandit methods \cite{DBSSZ07icdm} provide the best strategies.
In \cite{LMALS14ecml}, a bandit-based active learning algorithm is proposed for experiment design in dynamical system identification.
These approaches are directly relevant to biological experiment design and modelling. %and are now part of the DREAM challenge  \cite{Meyer14bmc}.
They should gain importance in the forthcoming years with the increasing automation of biological experiments.





\section{Conclusion and perspectives}
\wip{}

\bibliographystyle{siam}
\bibliography{biblio,contraintes}

\end{document}
