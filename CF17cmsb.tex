\documentclass{llncs}
\pagestyle{plain}
\usepackage{amsmath,amssymb,amsfonts,stmaryrd}
\usepackage{graphicx}
\usepackage[usenames,dvipsnames]{color}
\usepackage{hyperref}
\usepackage[T1]{fontenc}
\usepackage{verbatim}
\usepackage{listings}
\lstset{
   basicstyle=\ttfamily, %\scriptsize\ttfamily,
%   frame=single,
   breaklines=true,
}
%\usepackage{algorithm}
\usepackage{float}

\newcommand{\wip}[1]{\textcolor{Purple}{WIPWIPWIPWIP #1 WIPWIPWIPWIP}}
\newcommand{\francois}[1]{\textcolor{blue}{#1}}
\newcommand{\sylvain}[1]{\textcolor{green}{#1}}

\newcommand{\defleq}{\sqsubseteq_{\text{def}}}
\newcommand{\lra}{\longrightarrow}

\floatstyle{ruled}
\newfloat{algorithm}{h}{algfig}
\floatname{algorithm}{Algorithm}

\begin{document}

\title{Probably Approximately Correct Learning of Regulatory Networks from Time-Series Data}

\author{Arthur Carcano\inst{1} \and Fran\c{c}ois Fages\inst{2} \and Sylvain
Soliman\inst{2}}

\institute{%
Ecole Normale Sup\'erieure, Paris, France\\
  \email{arthur.carcano@ens.fr}
\and Inria, University Paris-Saclay, Lifeware group, France\\
   \email{Francois.Fages@inria.fr},
   \email{Sylvain.Soliman@inria.fr}
}

\maketitle

\begin{abstract}
Automating the process of model building from experimental data 
is a very desirable goal to palliate the lack of modellers for many applications.
Despite the spectacular progress of machine learning techniques in data analytics, classification and clustering,
learning dynamical models from data time-series is challenging.
In this paper we investigate the use of the Probably Approximately Correct (PAC) learning 
framework of Leslie Valiant as a method for the automated discovery of influence models of biochemical processes from Boolean and stochastic traces. 
We show that Thomas's Boolean influence systems can be naturally represented by k-CNF formulae
and learned
from Boolean transition samples with a quasi linear number of Boolean transition samples,
and that positive Boolean influence systems can be represented by monotone DNF formulae.
We evaluate PAC learning algorithms in this context on Boolean models of T-lymphocyte and MAPK signalling,
and discuss the merits of this framework as well as its limitations with respect to realistic experiments.
\end{abstract}

\section{Introduction}

Modelling biological systems is still an art which is currently limited in its applications by the number of available modellers.
Automating the process of model building is thus a very desirable goal
to attack new applications, develop patient-tailored therapeutics,
and also design experiments that can now be largely automated
with a gain in both the quantification and the reliability of the observations, at both the single cell and population levels.

Machine learning is revolutionising the statistical methods in biological data analytics,
data classification and clustering, and prediction making.
However, learning dynamical models from data time-series is more challenging.
There has been early work on the use of machine learning techniques, such as inductive
 logic programming~\cite{Muggleton95ngc} combined with active learning in the vision of the ``robot scientist''~\cite{BMOKRK01etai},
to infer gene functions,
metabolic pathway descriptions~\cite{AM02etai,AM02slps}
or gene influence systems~\cite{BCRG04jtb},
or to revise a reaction model with respect to CTL properties~\cite{CCFS06tcsb}.
Since a few years, progress in this field can be measured on public benchmarks
of the ``Dream Challenge'' competition~\cite{Meyer14bmc}.
Logic Programming, and especially \emph{Answer Set Programming} (ASP), provide efficient tools such as CLASP~\cite{GKNS07lpnmr}
to implement learning algorithms for Boolean models.
They have been applied in~\cite{GSTUV08iclp} to the detection of  inconsistencies in large biological networks,
and have been subsequentially applied to the inference of gene networks from gene expression data and to the design of discriminant experiments \cite{VKASSSG15frontiers}.
Furthermore, ASP has been combined with CTL model-checking in~\cite{OPSSG16biosystems} to learn mammalian signalling networks from time series data,
and identify erroneous time-points in the data.

Active learning extends machine learning with the possibility to call oracles, e.g.~make experiments,
and budgeted learning adds costs to the calls to the oracle.
The original motivation for the budgeted learning protocol came from medical applications in which the outcome of a treatment,
drug trial, or control group is known, and the results of running medical tests are each available for a price~\cite{DZBSM13ml}.
In this context, multi-armed bandit methods~\cite{DBSSZ07icdm} currently provide the best strategies.
In~\cite{LMALS14ecml}, a bandit-based active learning algorithm is proposed for experiment design in dynamical system identification.

In this paper, we consider the framework of Probably Approximately Correct (PAC) Learning 
which was introduced by Leslie Valiant in his seminal paper on a theory of the learnable~\cite{Valiant84cacm},
Valiant questioned what can be learned from a computational viewpoint,
and introduced the concept of probably approximately correct (PAC) learning,
together with a general-purpose polynomial-time learning protocol.
Beyond the learning algorithms that one can derive with this methodology,
Valiant's theory of the learnable has profound implications
on the nature of biological and cognitive processes,
of collective and individual behaviors,
and on the study of their evolution~\cite{Valiant13book}.

In this paper, we investigate PAC learning as a method for the automated discovery of influence models of biochemical processes from time-series data. 
To the best of our knowledge, 
the application of PAC learning to models of biological systems has not been reported before.
We show that Thomas's gene regulatory networks \cite{Thomas91jtb,Thomas73jtb} can be naturally represented by 
Boolean formulae in conjunctive normal forms with a bounded number of litterals (i.e.~k-CNF formulae),
and can be learned from Boolean transition samples with a quasi linear number of Boolean transition samples, using Valiant's PAC learning algorithm for k-CNF formulae.
We also show that Boolean influence systems with their positive Boolean semantics discussed in \cite{FMRS16cmsb}
can be naturally represented by monotone DNF formulae.
Valiant's PAC active learning algorithm for monotone DNF formulae makes it possible in theory 
to learn a Boolean influence model from a set of positive examples and a set of calls to an oracle for the Boolean transitions.
However, ...

These results are illustrated in the following with a Boolean influence model of the lymphocyte T as running example.
They are further evaluated on... of size ...
We evaluate the PAC learning protocols first without prior knowledge, and then with knowledge of the unsigned or signed influence graph without Boolean functions.

\sylvain{will be completed when we know what is in the article or not}

We conclude on the merits of this framework, but also on its limits to scale up,
and to be usable in the context of a wet lab context for real biological experiment design.


\section{Preliminaries on PAC Learning}\label{pac}

\subsection{PAC Learning Protocol}

\francois{Ouh la on ne peut pas changer les definitions de Valiant comme ca sans explliquer,
il faut reprendre la terminologie de Valiant vecteurs dans {0,1,*} vecteurs totaux etc.}

Let us consider a finite set of Boolean variables $x_1,\ldots,x_n$,
\begin{itemize}
	\item A vector is an assignment of the $n$ variables to $\mathbb{B} = \{0,1,*\}$
	\item A total vector is an assignment in $\{0,1\}$
	\item A Boolean function $F:{\mathbb{B}}^n \rightarrow \mathbb{B}$
	assigns a Boolean value to each total vector.
\end{itemize}

The idea behind the PAC learning protocol is to discover a hidden Boolean function $F$ while restricting oneself to the two following operations~:
\begin{itemize}
  \item
\textsc{Sample}$()$~: which returns a positive example, i.e.~a total vector $v$ such that $F(v)=1$. 
The output of \textsc{Sample}$()$ follow a given probability distribution $D(v)$, which will be later used to measure the approximation of the result.
  \item
\textsc{Oracle}$(x)$~: which calls an oracle on some input $v$ to obtain the value of $F(v)$
\end{itemize}


\begin{definition}[Learnable class~\cite{Valiant84cacm}]
	\label{def:learnclass}
   A class $\cal M$ of \emph{Boolean functions} is said to be \emph{learnable}
   if there exists an algorithm $\cal A$ such that:
   \begin{itemize}
      \item $\cal A$ runs in polynomial time in $n$ -- the dimension of the models to learn -- and $h$ the precision parameter.
      \item
         For all functions $F$ in $\cal M$, and all distributions $D$ on the positive examples outputted by \textsc{Sample}$()$,
         $\cal A$ deduces with probability higher than $1-h^{-1}$ an approximation $G$ of $F$ in $\cal M$ such that
         \begin{itemize}
            \item $G(v)=1$ implies $F(v)=1$ (no false positive)
            \item
               $\displaystyle\sum_{v\ s.t.\ F(v)=1\wedge G(v)=0} D(v) < h^{-1}$ (low probability of false negatives)
         \end{itemize}
   \end{itemize}
\end{definition}


\subsection{PAC Learning Algorithms}

Valiant showed the learnability of some important classes of functions in this framework,
in particular for Boolean formulae in conjunctive normal forms with at most $k$ literals per conjunct (k-CNF),
and for monotone (i.e.~negation free) Boolean formulae in disjunctive normal form (DNF).

The computational complexity of the PAC learning algorithms for these classes of functions is expressed in terms of the function
$L(h,S)$ defined as the smallest integer $i$ such that
in $i$ independent Bernoulli trials, each with probability at least $h^{-1}$ of success, the probability of having fewer than $S$ successes is less than $h^{-1}$.
Interestingly, this function is quasi-linear in $h$ and $S$, i.e.~for all
integers $S\ge 1$ and reals $h>1$, we have $L(h,S) \le 2h(S+\log_e h)$~\cite{Valiant84cacm}.

\begin{theorem}[\cite{Valiant84cacm}]\label{thm:kcnf}
For any $k$, the class of $k$-CNF formulae on $n$ variables is learnable with an
algorithm that uses $L(h,{(2 n)}^{k+1})$ examples and no oracle.
\end{theorem}

The proof is constructive and relies on Alg.~\ref{algCNF} below. In this algorithm, the initialization of the learned function $g$ to the false constraint expressed as the conjunction of all possible \emph{clauses} (i.e.~disjunctions of litterals)
leads to the learning of a minimalist  generalization of the positive examples with mainly no false positive and low probability of false negatives.

\begin{algorithm}
\begin{enumerate}
  \item Initialise $g$ to the conjunction of all possible clauses of at most $k$ literals (there are $(2n)^k$ such clauses),
\item Call $L(h,(2n)^{k+1})$ positive examples $v$, and for each $v$~:
\begin{itemize}
\item Delete all the clauses in $g$ that do not contain a literal true in $v$.	
\end{itemize}
\item Output: $g$
\end{enumerate}
\caption{PAC-learning of $k$-CNF formulae.\label{algCNF}}
\end{algorithm}


Let the \emph{degree} of a Boolean formula be the largest number of prime
implicants in an equivalent rewriting of the formula as a non-redundant sum of
prime-implicants.


\begin{theorem}[\cite{Valiant84cacm}]\label{thm:mdnf}
    The class of monotone DNF formulae on $n$ variables is also learnable with an
    algorithm that uses $L(h,d)$ examples and $d n$ calls to the oracle,
    where $d$ is the \emph{degree} of the function to learn~\cite{Valiant84cacm}.
\end{theorem}

The proof relies on Alg.~\ref{algDNF}. As previously, the algorithm guarantees that a minimal generalization is learned, up to the approximations previously defined.

\begin{algorithm}
\begin{enumerate}
\item Initialise $g$ with false (constant zero),
\item
Do $L(h,d)$ calls to positive examples $v$, and for each $v$,
\begin{itemize}
	\item 
	If $g$ is not implied by $v$, add monomial $m$ to $g$, where $m$ is the conjunction
   of literals determined in $v$ that are essential to $f$ (the function being
   learned).
\item \francois{to describe with more details and logical notations as in Valiant 84}
\end{itemize}
\item Output: g
\end{enumerate}
\caption{PAC-learning of monotone DNF formulae.\label{algDNF}}
\end{algorithm}

The polynomial computational complexity follows from the fact that each monomial $m$ is a prime implicant
of $f$ by construction, and that it is constructed by at most $n$ calls to the
oracle.

\subsection{Implementation}

In our implementation of the PAC-learning algorithm for $k$-CNF formulae,
we make use of the lattice structure of $k$-clauses ordered by implication.
This data structure allows
\begin{itemize}
	\item $O(1)$ access to any $k$-clause
	\item From clause $a$, $O(1)$ access to the smallest clauses implied by $a$ and to the biggest clauses that imply $a$
\end{itemize}



\section{Boolean Models of Molecular Regulatory Networks}

In this section, we present four Boolean formalisms used to model regulatory networks in cell molecular biology.
%and to which we will later whether or not PAC learning is applicable.
We assume a finite set of molecular species $\{x_1,\dots,x_n\}$ 
and consider Boolean states that represent the activation or presence of each molecular species of the system, 
i.e.~total vectors in $\mathbb{B}^n$ that specify whether or not the $i$th species is present, or the $i$th gene activated.

\subsection{Influence Systems with Forces}

\francois{I tried to import our definitions but the stochastic-Boolean semantics used in this paper is a bit different...}

Influence systems with forces have been introduced in~\cite{FMRS16cmsb}
to generalize the widely used logical models of regulatory networks \emph{\`a la} Thomas~\cite{Thomas73jtb} 
in order to provide them with a hierarchy of semantics
including differential, stochastic, Petri Net and Boolean semantics.

\begin{definition}\cite{FMRS16cmsb}
An \emph{influence system}
   $I$ is a set of quintuples $(P, I, t, \sigma, f)$ called \emph{influences},
   where 
\begin{itemize}
\item $P$ is a multiset on $S$, called \emph{positive sources} of the influence, 
\item $I$ a multiset of \emph{negative sources}, 
\item $t\in S$ is the \emph{target},
\item $\sigma\in\{+,-\}$ is the \emph{sign} of the influence, accordingly called either \emph{positive or negative influence},
\item and $f:\mathbb{R_+}^n\to\mathbb{R}$, called the \emph{force} of the influence,
is a partially differentiable function, non-negative
   on $\mathbb{R}_+^n$;
\item $x_i\in P$ if and only if $\sigma = +$ (resp.\ $-$) and
   ${\partial {f}}/ {\partial x_i}(\vec x)>0$ (resp.\ $<0$) for some value
   $\vec x\in\mathbb{R}_+^n$;
\item $x_i\in I$ if and only if $\sigma = +$ (resp.\ $-$) and
   ${\partial {f}}/ {\partial x_i}(\vec x)<0$ (resp.\ $>0$) for some value
  $\vec x\in\mathbb{R}_+^n$;
\end{itemize}
\end{definition}

Influences of sign $+$ are called {positive influences} and those of
sign $-$, {negative influences}.
In addition, we distinguish the positive sources from the negative sources of an influence (positive or negative),
in order to annotate the fact that in the differential semantics,
the source increases or decreases the force of the influence,
and in the boolean semantics with negation whether the source or the negation of the source
is a condition for a change in the target.

In the examples below, we use the ASCII syntax of Biocham v4 for influences.
Positive (resp.~negative) influences are written with an arrow \lstinline|->| (resp.~\lstinline+-<+)
which separates the sources from the target.
The positive and negative sources are separated
by a \lstinline|/|, which can be omitted if there is no negative source.

\begin{example}\label{ex:LVi}
   The classical birth-death model of Lotka--Volterra can be represented by the following
influence system between a proliferating prey $A$ and a predator $B$:
  \lstinputlisting[firstline=6,lastline=9]{examples/LVif.bc}
   composed of four influences with no negative sources:
   $(\{A, B\}, \emptyset, A, -, k1*A*B)$, $(\{A, B\}, \emptyset, B, +, k1*A*B)$, $(\{A\}, \emptyset, A, +, k2*A)$ and $(\{B\}, \emptyset, B, -, k3*B)$.

This example contains both positive and negative influences but no influence inhibitor, i.e.~no negative source in the influences.
For an example of influence with inhibitor, one can consider the specific inhibition of the proliferation rate of $A$ by some variable $C$
which is distinguished from a general negative influence of $C$ on $A$, by writing $C$ as an inhibitor of the positive influence of $A$ on $A$:
   \begin{lstlisting}
k2*A/(1+C) for A/C -> A.
   \end{lstlisting}
\end{example}

While it may seem natural in the Boolean semantics of an influence system to interpret the negative sources by negations
for the enabling conditions for applying an influence in a given state, 
the approximation and abstraction relationships that link the different semantics of a given influence system,
lead us to consider the positive semantics of influence systems which simply ignores the negative sources~\cite{FMRS16cmsb}.
Here, we adopt the Boolean semantics with negation and consider the particular case of \emph{positive influence systems}
which may contain positive and negative influences but only positive sources no negative sources.


\begin{definition}[Boolean Semantics]
	The Boolean semantics of an influence system $\{(P_i, t_i, \sigma_i, f_i)\}_{i\in I}$
	over a set $S$ of $n$ variables,
	is the Boolean transition system $\lra$ defined over Boolean state total vectors in $\mathbb{B}^n$
	by
	${\vec x}\lra{\vec x'}$ if there exists an influence $(P_i, t_i, \sigma_i, f_i)$
	such that ${\vec x}\models \bigwedge_{p\in P_i}$
	and ${\vec x'} = {\vec x}\ \sigma_i\ t_i$.
\end{definition}


\subsection{Monotone DNF Representation of Positive Influence Systems}


In this section we restrict ourselves to the positive semantics of influence systems
and syntactically to influence systems without inhibitors, as in Ex.~\ref{ex:LVi}.

\francois{to update}

\begin{definition}
	
	An \emph{influence system} (without inhibitor) $I$ on a set of variables $S=\{x_1,\dots,x_n\}$ is a
	finite set of quadruples $(P, t, \sigma, f)$, called \emph{influences}, where
	$P\subset S$ is called the set of \emph{positive sources} of the influence,
	%   $N\subset S$ \emph{negative sources}, 
	$t\in S$ is the \emph{target} of the influence,
	$\sigma\in\{+,-\}$ is its \emph{sign positive or negative}, and $f$ is a
	real-valued mathematical function $P\to\mathbb{R^+}$, called the
	\emph{force} of the influence.
	
\end{definition}

\begin{definition}[Positive Boolean Semantics]
	The positive Boolean semantics of an influence system $\{(P_i, t_i, \sigma_i, f_i)\}_{i\in I}$
	over a set $S$ of $n$ variables,
	is the Boolean transition system $\lra$ defined over Boolean state total vectors in $\mathbb{B}^n$
	by
	${\vec x}\lra{\vec x'}$ if there exists an influence $(P_i, t_i, \sigma_i, f_i)$
	such that ${\vec x}\models \bigwedge_{p\in P_i}$
	and ${\vec x'} = {\vec x}\ \sigma_i\ t_i$.
\end{definition}

Positive influence systems may hence be represented as $n$ activation and $n$ deactivation functions, which are defined as follows:


\begin{definition}
	\label{def:activation}
	Given a network of $n$ genes $x_1,\ldots,x_n$, and
	an integer $1 \leq k \leq n$ we can define ${x_k}^+$ (resp.\ ${x_k}^-$):
	${\{0,1\}}^n \rightarrow\{0,1\}$ the activation (resp.\ deactivation)
	Boolean function of $x_k$. In a given state $v$, ${x_k}^+(v)$ (resp.\ ${x_k}^-$(v)) is worth 1 if and only if the gene $x_k$ can be activated (resp. deactivated) from state $v$.
\end{definition}

The natural representation for (de)activation function and influences being the DNF one, and because we are here focusing on \textbf{positive} influence systems, monotone DNF seems like the perfect fit to represent positive influence system and study the applicability of PAC learning algorithms to it.

\begin{example}

For Ex.~\ref{ex:LVi} we would have as monotone DNF formulae:
\begin{eqnarray*}
   A^+&=&(A)\\
   A^-&=&(A \wedge B)\\
   B^+&=&(A\wedge B)\\
   B^-&=&(B)
\end{eqnarray*}

\end{example}

\subsection{$k$-CNF Representation of Influence Systems}

Yet, if one wishes to deal with influence systems containing inhibitors, the monotone DNF representation is no longer a fit.

Instead, one may switch to the $k$-CNF representation of the system, which constrains not the absence of inhibitors but the number of species that can play a given "role". To be more precise, let's consider the following CNF for a hypothetic influence:

\[
\left(a \vee b \vee c\right) \bigwedge
\left(d \vee e\right) \bigwedge 
\neg f
\]

Then each of the clause can be interpreted as a role, and for the activation of our hypothetic gene, we need at least one species that plays this role. For example, the enzyme can be $a$, $b$, or $c$ whereas 
the reactant is $d$ or $e$, and we need that $f$ be inactive.


\begin{example}
   The Lotka--Volterra example Ex.~\ref{ex:LVi} with inhibition cannot be
   translated to a monotone DNF formula. It has however the following CNF
   representation. It is easy to
   notice that $k=1$ is enough since there is only one positive and one
   negative influence for each target.
\begin{eqnarray*}
   A^+&=&(A)\wedge(\neg C)\\
A^-&=&(A) \wedge (B)\\
B^+&=&(A)\wedge (B)\\
   B^-&=&(B)
\end{eqnarray*}

\end{example}


\subsection{$k$-CNF Models of Thomas's Regulatory Networks}

Finally, Valiant's framework perfectly fits the Boolean semantics of Thomas's gene
regulatory networks~\cite{Thomas73jtb}.

\begin{definition}
   A \emph{Thomas} network is defined by a set of genes $\{x_1,\dots,x_n\}$
   and $n$ Boolean functions $\{f_1,\dots,f_n\}$ describing for each gene its
   possible next state, given the current state.
\end{definition}

$k$-CNF formulae can be used to represent such gene regulatory network functions with some reasonable restrictions on their connectivity.
In particular, it is worth noticing that in Thomas networks of degree bounded by $k$,
each gene has at most $k$ regulators, each gene activation function $f_i$ thus depends of at most $k$ variables
and can certainly be represented by a $k$-CNF formula.

As proven in~\cite{FMRS16cmsb}, in the Boolean setting any influence system is
also a Thomas' network.
As this will be useful later on, we also remark that the Thomas' Network can be computed from the (de)activation functions presented in definition~\ref{def:activation} by:

\[
\forall 1 \leq i \leq n, \forall v, f_i(v) = \left\{\begin{array}{l}
1 \text{ if } \left\{\begin{array}{l}
v_i = 0 \text{ and } {x_i}^+(v) = 1\\
v_i = 1 \text{ and } {x_i}^-(v) = 0 \\
\end{array}\right.\\[1em]
0 \text{ if } \left\{\begin{array}{l}
v_i = 0 \text{ and } {x_i}^+(v) = 0\\
v_i = 1 \text{ and } {x_i}^-(v) = 1\\
\end{array}\right.
\end{array}\right.
\]
\sylvain{manque tous les cas où les deux fonctions sont égales (surtout les
deux égales à 0, i.e. on reste sur place)}

\sylvain{manquerait peut-être une preuve}


\begin{example}
   We can apply the above translation to our Lotka--Volterra example of
   Ex.~\ref{ex:LVi}. We obtain:
   \begin{eqnarray*}
   f_A &=& A \wedge\neg B\\
   f_B &=& 0
   \end{eqnarray*}

   Note that the form of $f_B$ only means that the only possible state change
   for $B$ is from $1$ to $0$. Of course $B$ can also stay at $1$ but
   non-terminal self-loops do note appear in such a logical model.
\end{example}

\section{Experimental Setup}

%To further discuss the applicability of PAC learning to actual experiments, we hereby introduce what we believe is a plausible experimental setup, and discuss its implications on the learning of different classes of functions.

\subsection{Steps and Traces}

From now on, we do not assume %anymore 
that we have full access to the hidden Boolean function for
\textsc{Sample} and \textsc{Oracle}, but we restrict ourselves to the observations that can be obtained from data time-series, or traces,
produced either from real biological experiments, or, for the purpose of evaluating the learning method, from simulations.

\begin{figure}[htbp]
	{\large 
	\[
	\cdots
	\rightarrow
	\underset{\vspace{1em}(a)}{
		\left(\begin{array}{c}
		0\\ 1\\ 0\\ 1
		\end{array}\right)}
	\rightarrow
	\underset{\vspace{1em}(b)}{
		\left(\begin{array}{c}
		\textbf{\textit{1}}\\ 1\\ 0\\ 1
		\end{array}\right)}
	\rightarrow
	\underset{\vspace{1em}(c)}{
		\left(\begin{array}{c}
		1\\ 1\\ 0\\ \textbf{\textit 0}
		\end{array}\right)}
	\rightarrow
	\cdots
	\]
}
	\caption{\label{steps}Illustration of our hypothetical experimental setup with three steps of a trace. Between $a$ and $b$, the first gene has been activated, and between $b$ and $c$, the last one has been deactivated.}
\end{figure}

More specifically, and as illustrated in figure \ref{steps}, we consider that our experimental setup enables us to identify a "step" of its evolution, i.e.~that we are able to obtain a trace $(v_i)_{0 \leq t \leq T}$ of the boolean state of activation of its species where for all $t$ in $[0,T-1]$, $v_t$ and $v_{t+1}$ differ in exactly one coordinate. That is, the state of exactly one species has changed.


Even though we will finally not require it, we first also consider that our experimental setup enables us to put the system in a given state, i.e.~ that for any $v \in \mathbb{B}^n$ we can set the experiment's initial conditions so that they are correctly abstracted by $v$.

\subsection{Application of PAC Learning on Transition Traces}
%Now that our hypothetical experimental setup is described, we will discuss how the two key operations of PAC-learning (\textsc{Sample} and \textsc{Oracle}) can be implemented, if at all, in it.

Because both influence systems and \textit{\`{a} la Thomas} networks essentially boil down to (de)activation functions we limit ourself to those in the following.

\subsubsection{Sampling}

We can first remark that sampling of (de)activation function is straightforward in this setting. Referring to figure~\ref{steps}, $(a)$ is a positive example for the activation function ${x_1}^+$, and $(c)$ for ${x_4}^-$.

A call to \textsc{Sample} can then simply be a search in the trace for the next positive example for the current function. Issues \wip{that have not been studied yet} may arise if there is not enough samples to guarantee a good approximation. 
%Our belief is 
We assume here that those issues can be solved by running several traces from different initial states.

\subsubsection{Oracle}
The \textsc{Oracle} function needs to evaluate the (de)activation function on a given total vector $v$, that is, it needs to be able to set the system in a state abstracted by $v$ and say whether or not a given gene can be (de)activated from this state.

The intuitive solution is the following : set the system in the desired state and see whether or not the gene is (de)activated. 
However, different atomic steps are possible from a given state and we have no guarantee that the one we are interested in will happen
in several runs of the experiment. 

Moreover, in its simple form presented above, %the current state of 
Valiant's framework does not allow for the oracle to be probabilistic.
In practice however, since we cannot run this "state and observe the first atomic step" scheme an infinite number of time to be able to tell that the (de)activation can only happen with probability 0,
we have to content ourself with an imperfect oracle.
%A peculiarly fast-thinking reader may also have notice that different (and often infinitely many) states of the system may also be abstracted by the same boolean total vector $v$, and that the transition may also happen from only some of them.
%For those reasons, algorithm requiring the \textsc{Oracle} function seem not applicable as of today, and we choose to only focus on $k$-CNF and sampling from now on.


%\section{In sillico application}
\section{PAC Learning from Boolean Traces}

\francois{do not use figures for code}

\francois{please pretty print [[]] with /\ \/ }

A first experimentation was to simulate boolean traces for a given influence network, and use them as a basis to learn. A toy example is given in figure \ref{bool-LV} (influence network) and the correspondign results are presented in figures \ref{bool-LV.res} and \ref{bool-LV.res.pretty}

Output is to be read as follow:\\
\texttt{Foo+:~[['A','Z'],~['!E'],~['!Foo']]} means that the activation (\texttt{+}) function of B is $(A \vee Z)\wedge\neg E$. Remark that to be activated, Foo obviously needed to be deactivated first.

\subsubsection{Results}

\begin{figure}[htp]
	\verbatiminput{examples/bool-lokta.reac}
	\vspace{-1em}
	\caption{An influence system describing the Lokta-Voltera prey vs. predator model. Numbers in parenthesis indicate the force of an influence. \label{bool-LV}}
\end{figure}
\begin{figure}[htp]
	\verbatiminput{examples/bool-lokta.res}
	\vspace{-1em}
	\caption{Results of PAC-learning on traces of the Boolean simulation of the Lokta-Voltera toy example.\label{bool-LV.res}}
\end{figure}
\begin{figure}[htp]
	\ttfamily
	Predator+: $False$
	
	Predator-: Predator $\wedge$ (no Prey)
	
	Prey+: $False$
	
	Prey-: Predator $\wedge$ Prey
	\rmfamily
	\caption{Prettified results of PAC-learning on traces of the Boolean simulation of the Lokta-Voltera toy example.\label{bool-LV.res.pretty}}
\end{figure}
\pagebreak
The results can be interpreted as follow:

Both the predator and the prey species cannot appear. It is important to note that the activation functions in the Lokta Voltera models means the apparition on extinction of the species as a whole and not of individuals of it.

For the predator to disappear, it is necessary that there is predator in the first place and that there is no prey. If the first part of this conjunction is obviously true, the second is false: strictly speaking, predators may disappear even if there is prey left, yet this case is very unlikely: the most likely case is that the predator will go extinct only once there are no more preys left for it to eat. This is a good example of how the learning is only approximate.

Finally, for the prey to go extinct, there must be both prey in the first place and predator to eat it. This is true.

\subsubsection{On the quantification of the approximation}

As it can be seen even on this very simple example, the "approximately" in "probably approximately correct" is not there for nothing. Yet, as explained in definition \ref{def:learnclass}, the quantification of this approximation relies on the knowledge of the distributions of the samples.

Even though we were not able to write it down exactly, it is our conjecture that in the present case, the probability of a positive example $v$ of (de)activation function $x\pm$ to be sampled is strongly and intuitively correlated to both the probability that the system reaches state $v$ and the probability of the actual (de)activation of gene $x$ from state $v$. 



\section{PAC Learning from Stochastic Traces}

In order to push our experiments even further, we shall now produce the trace
data, not directly from a Boolean model, but from more realistic simulations.

For the sake of evaluating the learning algorithm,
we simulate wet lab experiments \emph{in silico}, using Gillespie's algorithm for stochastic simulation.
This stochastic semantics gave us traces in ${\mathbb{N}}^n$ that we abstract as traces in ${\{0,1\}}^n$.

%Our simulator uses a custom language to specify experiments and allows us to set the amount of each species before the experiment is run. The number of run is given by Valiant's bound on the number of sampling calls.


We ran two example reactions, given in figures~\ref{test}, and \ref{preypred}.

\begin{figure}[htbp]
	\verbatiminput{examples/test.reac}
	\vspace{-1em}
	\caption{A test reaction, A and E appear naturally in the medium, and A can be turned into B in absence of E. B  can be turned into C. All of the species can disappear due to dilution.\label{test}}
\end{figure}
\begin{figure}[htbp]
	\verbatiminput{examples/lokta.reac}
	\vspace{-1em}
	\caption{A Prey-Predator model. The first line indicates the starting quantities for each species. This corresponds to the influence model given in figure \ref{bool-LV}.\label{preypred}}
\end{figure}

The first example gave perfect results, as indicated on figure~\ref{test_res}.
\begin{figure}
	\verbatiminput{examples/test.result}
	\caption{Results for the test example\label{test_res}}
\end{figure}

As previously, the prey-predator example (figure~\ref{preypred_res}) displays a typical error: because there is more than 1 prey, in the Gillespie algorithm, eating of a prey by a predator is way more likely than the death of old age of a predator, and hence the deactivation (ie extinction) of predators happen only when they cannot eat anymore and the only possible reaction is the third: death. This lead the learning algorithm to believe that the absence of prey is needed for extinction.
\begin{figure}
	\verbatiminput{examples/lokta.result}
	\caption{Results for the Prey-Predator model\label{preypred_res}}
\end{figure}

\section{PAC Learning with Prior Knowledge}

\francois{Will be much clearer with some drawing of influence graphs}

\begin{example}
   \sylvain{I will add some introductory text here to present the example}
   The actual code of this model is displayed in Fig.~\ref{lympho}.
\begin{figure}[htbp]
	\verbatiminput{examples/lympho-bool.reac}
	\caption{The Th lymphocyte differentiation model of~\cite{RRMTC06tcsb}.\label{lympho}}
\end{figure}
\end{example}

All the systems proposed so far were of humble size. An actual influence system of bigger dimensions is the one that governs the differentiation of the lymphocyte T, as described by R\'{e}mi, Ruet and Thieffry\cite{RRMTC06tcsb}. Its description is given figure \ref{lympho} (page \pageref{lympho}).

However, results of learning on the traces resulting from the boolean simulation of this system gave very poor results (figure \ref{lympho_res}). Thankfully, the PAC learning algorithm for $k$-CNF can be modified to be able to take hints into account.

\wip{explain hints}

After supplying hints indicated in figure \ref{hints}, the results (figure \ref{hints.res}) are \wip{good}.
	
\begin{figure}
	\verbatiminput{examples/lympho_edited.result}
	\caption{Results for lymphocite model, edited. Lines with MISS are one on which the algorithm is too far. The tautological negation (\texttt{!A} for \texttt{A+}) have also been deleted.\label{lympho_res}}
\end{figure}

However, when given hints on the possible relations, as indicated in
Fig.~\ref{hints}, results (given Fig.~\ref{res_hints}) are correct.

\begin{figure}
	\verbatiminput{examples/lympho.hints}
	\caption{Hints for the lymphocyte model. For each species, a set of possible influencers is given. The PAC algorithm will then learn a model in which only the specified influencers can either induce or inhibit the species.\label{hints}}
\end{figure}
\begin{figure}
	\verbatiminput{examples/lympho-bool.res}
	\caption{Results for lymphocite model, with hints.\label{hints.res}}
\end{figure}


\section{Active Learning of Positive Influence Systems}
\label{sec:oracles}

\sylvain{I guess this is where the oracle as experiment-design discussion will
take place\dots}

The (positive) Boolean semantics of biochemical influence systems
can be directly represented by the disjunction of the (positive) enabling conditions of each, either positive or negative, influence on a given target,
i.e.~by a monotone DNF formula for each activation or inhibition of each target.
In the Lotka--Volterra influence system, the algorithm above is thus expected to learn the structure of the influence system
(without the stochiometry of course),
from the observation that the prey can disappear only in presence of the predator
while the predator can always disappear in presence or absence of the prey.

  Learning reaction models from observed transitions is much more tricky,
  since some reactions may change the Boolean value of several reactants or products in one single transition.
%  Let us first remark that the transition relation $r(x_1,\ldots,x_n,x'_1,\ldots,x'_n)$
%  is naturally represented in DNF by the disjunction of the conjunctions associated to each reaction
%  (remember that there can be several Boolean transitions associated to one reaction for taking
%into account the possibly partial or total consumption of the reactants).
%  is monotonic in the predecessor variables $x_i$'s in the postive Boolean semantics of reactions.
%  Furthermore it is also monotonic in the successor variables $x'_j$ since we consider
%  all partial or total consumptions of reactants.
  Therefore, it is not only the activation and inhibition functions of each species which are to be learnt,
  but the update functions of pairs and triples of species if we restrict to elementary reactions with at most two reactants or products.
  In this case, the update functions can be represented by monotonic DNF formulae, since the (positive) Boolean semantics of a reaction system does not test the absence.
Furthermore,   one cannot expect to learn the structure of such a reaction network
from the observation of the state transitions from one single initial state.
The learning algorithms assumes that the positive examples of the state transition relation be distributed
among the whole vector space.
For instance, in the MAPK example, in addition to the initial state of the wild type organism where all the kinases and phosphatases are present,
it is necessary to consider some mutated organisms, in which some kinases or phosphatases are absent,
in order to gain information on the precise conditions of activation and deactivation of the different forms of the kinases.
This strategy is essentially similar to what the biologists do to elucidate the structure of biological processes
in a qualitative manner.


\section{Conclusion}

In conclusion, we have shown that Valiant's work on PAC learning provides an elegant trail to serve as a way to automatically discover possible regulatory networks of a biological process, given sufficiently precise traces of its run.
When dimension increases, PAC learning algorithms can also leverage available prior knowledge on the system to deliver results.
Moreover, the use of oracles for monotone DNF formulae is of interest to create an online learning algorithm
and is relevant to experimental design.

More work is needed however to make comparisons on common benchmarks
with  other approaches already investigated in this context, such as Answer Set Programming (ASP) and budgeted learning,
and to investigate the applicability to real experiments with particular biological technology.

\bibliographystyle{splncs03}
\bibliography{contraintes}

\end{document}
